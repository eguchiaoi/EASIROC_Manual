
\section{How to Use EASIROC}
\subsection{Trigger signals}
Basically, only external trigger can be used. When you want to use self trigger, use output signals with appropriate delay. In that case, you should enter the signal with the order; "HOLD", "T STOP", "ACCEPT". 
You should insert more than about 2 us between HOLD and ACCEPT. It depends on ADC converting rate but 2.5 us is maybe enough.

\begin{figure}[H]
\begin{center}
\includegraphics[width = 13.0cm, bb= 0 0 899 358]{1.png}
\end{center}
\caption{An example of self trigger circuit[1]}
\label{fig:}
\end{figure}

\begin{figure}[H]
\begin{center}
\includegraphics[width = 1.3cm, bb= 0 0 588 3683]{4.jpg}
\end{center}
\caption{The front panel of EASIROC board}
\label{fig:}
\end{figure}

\newpage
\subsection{Interactive mode}
EASIROC has an inner read line so you can handle EASIROC module interactively.

\subsubsection{Handle interactive mode}
Start interactive mode.
\begin{shadebox}
\begin{verbatim}
$ ./Controller.rb [IP Address (default: 192.168.10.16)]
\end{verbatim}
\end{shadebox}
 \\
Stop interactive mode.
\begin{shadebox}
\begin{verbatim}
> exit [quit]
\end{verbatim}
\end{shadebox}


\subsubsection{Set bias voltage}

Print the voltage and current value of HV.
\begin{shadebox}
\begin{verbatim}
> statusHV
\end{verbatim}
\end{shadebox}
 \\
Set the HV at the value of [bias voltage].
\begin{shadebox}
\begin{verbatim}
> setHV [bias voltage]
\end{verbatim}
\end{shadebox}
 \\
Increase HV with some steps. In each step, check the current and stop if it reaches to the limit.
\begin{shadebox}
\begin{verbatim}
> increaseHV [bias voltage]
\end{verbatim}
\end{shadebox}

\subsubsection{Data taking}
Reset the slow control values.
\begin{shadebox}
\begin{verbatim}
> slowcontrol
\end{verbatim}
\end{shadebox}
 \\
Start data aquisition.
\begin{shadebox}
\begin{verbatim}
> read [Event #] [Filename]
\end{verbatim}
\end{shadebox}
 \\
ON/OFF the ADC [TDC/Scaler].
\begin{shadebox}
\begin{verbatim}
> adc [on/off]
> tdc [on/off]
> scaler [on/off]
\end{verbatim}
\end{shadebox}
It makes data taking faster if you off the ADC [TDC/Scaler] when you don't need them.

\subsubsection{Other information}
\begin{itemize}
\item 対話時に入力されたコマンドは CommandDispatcher クラスによって処理される
\item 対話モードでhelp と入力してヘルプが見られる
\item 各コマンドは変数 COMMANDS に含まれているものが利用可能で、それぞれのコマンドはメソッドによって処理される
\item シェルのコマンドも一部使えるようになっている(ex. ls, mv, root)
\item それらは変数 DIRECT\_COMMANDS に含まれているものが利用可能
\item Controller ディレクトリ内で hist.cc を make してプログラムを生成していれば、read 終了後に自動でヒストグラムを生成する。出力先は Controller/data ディレクトリ内。
\end{itemize}

\newpage
\subsection{Slow Controll}
Slow Controll は yaml ディレクトリ下の YAML ファイル(拡張子:yml)によって指定される。\\
YAML は構造化されたデータを表現するフォーマットで、XML と似ているが YAML の方が人間にとって理解しやすい形式になっている。よく使いそうなものを以下に紹介する。

\subsubsection{RegisterValue.yml}

\begin{shadebox}
\begin{verbatim}
EASIROC1:  # Set for each chip
         Capacitor HG PA Fdbck: 100fF  # Define amplifying level. Capacitor values
         Capacitor LG PA Fdbck: 100fF  # are contained in RegisterValueAlias.yml
         Time Constant HG Shaper: 100ns  # Time constant of slow-shaper
         Time Constant LG Shaper: 50ns
         DAC code: 600  # Threshold of the discriminator after fast-shaper
 
EASIROC2:
         Capacitor HG PA Fdbck: same
         Capacitor LG PA Fdbck: same
         Time Constant HG Shaper: same
         Time Constant LG Shaper: same
         DAC code: same
 
High Gain Channel 1: 0   # Read out channel from HG1/HG2
High Gain Channel 2: -1  # 0 : read out || 1 : not read out
Probe Channel 1: -1      # Output from the probe of front panel
Probe Channel 2: -1
Probe 1: Out_fs
Probe 2: Out_fs  # Out_PA_HG, Out_PA_LG, Out_ssh_HG, Out_ssh_LG, Out_fs
SelectableLogic:
         Pattern: Or64        # OneCh_#, Or32u, Or32d, Or64, Or32And,...
         HitNum Threshold: 4  # Threshold for each OR logic. 0~64. Default: 0
         And Channels: -1     # Cannels used in And Logic. 0~63. Default: -1
TimeWindow: 4095ns
UsrClkOut: "OFF"  # Periodic signal from syn out of front panel # "ON", 1Hz,...
Trigger:          ## This "Trigger" values are not used for this version.
         Mode: 0  #0-7
         DelayTrigger: -1   #500MHz #default:-1, 0-253 #trig -> hold -> l1 -> l2
         DelayHold: -1      #25MHz
         DelayL1Trig: -1    #6MHz
         Width: raw
\end{verbatim}
\end{shadebox}

When increase the DAC of discriminator, threshold get lower.

\subsubsection{InputDAC.yml}
There written the 8-bit Input DAC values for 32 channels by chip. They can be changed between 256-511 (The top bit is always set to 1 (=enable)).\\
When increase the DAC value, bias voltage get lower. 

\begin{shadebox}
\begin{verbatim}
---
EASIROC1:
  Input 8-bit DAC:
  - 350
  - 350
  - 350
  - 350
  - 350
  - 350
  - 350
  - 350
\end{verbatim}
\end{shadebox}

\subsubsection{Calibration.yml}

\begin{shadebox}
\begin{verbatim}
HVControl:   # Coefficient for converting the HV into DAC
        - 413.9 #423.06 #483.183
        - 747.8 #767.17 #780.0
MonitorADC:  # Coefficient for converting Monitor ADC values
             # into voltage, current and temperature
        HV: 0.00208 #0.3235 #0.00208
        HVOffset: 0.0355 #4.1694
        Current: 0.0364 #0.034
        InputDac: 0.00006866 #4.5/2^16 #0.0000685
        Temperature: 4500.0
\end{verbatim}
\end{shadebox}

\newpage
\subsection{Other information}

\subsubsection{Probe output}


\begin{multicols}{2}
信号処理中の中間信号を取り出すための Probe 出力ラインがフロントパネルに用意されている。出力することができる中間信号を以下に示す。
\begin{itemize}
\item HighGain PreAmp 出力
\item LowGain PreAmp 出力
\item HighGain Slow shaper 出力
\item LowGain Slow shaper 出力
\item Fast shaper
\end{itemize}
前述の RegisterValue.yml から出力する情報を指定することができる。\\

PreAmp を選択した場合、波形全体ではなくピーク付近の一部のみ出力される。注意点としては、2 ch 以上を同時に ON にしない、データ取得中は Probe 1,2 を OFF にすること。


\begin{figure}[H]
\begin{center}
\includegraphics[width = 3cm, bb= 0 0 602 1750]{6.jpg}
\end{center}
\caption{フロントパネルの Probe 出力}
\label{fig:}
\end{figure}
\end{multicols}


\begin{shadebox}
\begin{verbatim}
High Gain Channel 1: 0   # HG で読み出すチャンネルの指定
High Gain Channel 2: -1  # 読み出すなら ch_#、読み出さないなら-1
Probe Channel 1: -1      # Probe からの出力チャンネル選択
Probe Channel 2: -1      # 2 ch 以上同時に使用しない
Probe 1: Out_PA_HG
Probe 2: Out_fs  # Out_PA_HG, Out_PA_LG, Out_ssh_HG, Out_ssh_LG, Out_fs
\end{verbatim}
\end{shadebox}
 \\
また、SYNC OUT からの周期信号も RegisterValue.yml より変更が可能である。

\begin{shadebox}
\begin{verbatim}
UsrClkOut: "OFF"  # "OFF", "ON", 1Hz, 10Hz, 100Hz, 1kHz, 10kHz, 100kHz, 3MHz, ...
\end{verbatim}
\end{shadebox}


\newpage
\subsubsection{Yokoyama-lab's EASIROCs}
2019年5月9日現在、5つの EASIROC ボードの存在が確認された。うち一つは京大高エネルギー研究室の備品と思われる。IP アドレスは DAC との接続の際に必要となる。
\begin{table}[H]
\begin{center}
\caption{横山研所有のEASIROC}
\begin{tabular}{ccc} \hline
管理No. & IP Adress & 備考 \\ \hline
1 & 192.168.10.11 &  \\ 
2 & 192.168.10.12 &  \\ 
3 & 192.168.10.13 &  \\ 
2号 & 192.168.10.18 & ch.32 の読み出し不調。バイアス HV にふらつきあり。 \\ 
京大高エネ備品 & 不明 & 後半 32ch の InputDAC が動かない。連絡先:075-753-3837。  \\ \hline
\end{tabular}
\end{center}
\end{table}
